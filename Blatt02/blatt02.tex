\documentclass[a4paper]{article}
\usepackage[utf8]{inputenc}
\usepackage{amssymb}
\usepackage[ngerman]{babel}
\usepackage{hyperref}
\usepackage{enumitem}
\usepackage{listings}
\usepackage{esvect}
\usepackage{float}
\usepackage{graphicx}
\usepackage{xcolor}
\usepackage{todonotes}
\usepackage{pgfplots}
\usepackage{verbatim}
\pgfplotsset{compat=1.10}
\usepgfplotslibrary{fillbetween}
\usetikzlibrary{patterns}
\usepackage{mathtools}
\usepackage{centernot}

\hypersetup{
     colorlinks   = true,
     citecolor    = gray
}
\title{Grundlagen der Betriebssysteme\\ Blatt 01 \\ Gruppe 055}
\author{Marco Deuscher & Ibrahem Hasan}
\date{Mai 2019}

\begin{document}

\maketitle
\section{Festkomma Darstellung}
\paragraph{(a)}
\begin{align*}
    7,75_{10}=4+2+1+\frac{1}{2}+\frac{1}{4}=00111110_2
\end{align*}
Da die Zahl aus zweier Potenzen zusammengesetzt werden kann, ist sie exakt darstellbar.


\paragraph{(b)}
\begin{align*}
    2,71_{10}\approx2+\frac{1}{2}+\frac{1}{4}=00010110_2=2.75_{10}
\end{align*}
Die Zahl ist nicht exakt darstellbar. $|Z_{orginal}-Z_{Umwandlung}|=0,04$


\paragraph{(c)}
\begin{align*}
    5,375_{10} = 4+1+\frac{1}{4}+\frac{1}{8}=00101011_2
\end{align*}
Da die Zahl aus zweier Potenzen zusammengesetzt werden kann, ist sie exakt darstellbar.

\paragraph{(d)}
\begin{align*}
    9,12_{10}\approx=01001001_2=9,125_{10}
\end{align*}
Die Zahl ist nicht exakt darstellbar. $|Z_{orginal}-Z_{Umwandlung}|=0,005$

\section{Gleitkomma Darstellung}
Ergebnisse werden im folgenden in drei Gruppen unterteilt
\begin{enumerate}
    \item Vorzeichen
    \item Exponent
    \item Mantisse
\end{enumerate}
Außerdem ist der Bias mit $B=127$ gegeben. Es wird ein Vorzeichenbit verwendet, der Exponent hat eine Länge von 8bit und die Mantisse eine Länge von 23bit.
\paragraph{(a)}
\begin{align*}
    x=(-1)^v\cdot m \cdot2^{e-B} \Rightarrow m=\frac{x}{2^{e-B}}=1,109375
\end{align*}
Der Exponent ist gegegeben durch $127+4=131$ was in binär Darstellung wiederum $10000011$ entspricht. Durch Umwandeln Nachkommastellen der Mantisse in binäre Darstellung erhält man dann die folgende Darstellung

\begin{align*}
    0 \quad 10000011 \quad 00011100\dots 0
\end{align*}

\paragraph{(b)}
Analoges Vorgehen wie in der (a) liefert dann
\begin{align*}
    m=\frac{x}{2}=1,8125
\end{align*}
Für den Exponenten erhält man $127+1=128$. Durch Umwandeln in die binäre Darstellung erhält man dann
\begin{align*}
    0 \quad 10000000 \quad 110100\dots 0
\end{align*}


\section{Gleitkomma Operationen}
\todo[inline]{Noch zu bearbeiten}


\section{UTF8 Darstellung}
Hierbei ist die \color{red} Startsequenz, \color{cyan} Data, \color{olive} Beginn eines neues Bytes \color{black} jeweils markiert. 
\paragraph{(a)}
\begin{align*}
    \text{202e}=0010\;0000\;0010\;1110\\
    \color{red}1110\;\color{cyan}0010\;\color{olive}10\color{cyan}00\;0000\;\color{olive}10\color{cyan}10\;1110\\
\end{align*}


\paragraph{(b)}
\begin{align*}
    \color{red}11110\; \color{cyan} 000\; \color{olive}10\; \color{cyan}011111\; \color{olive}10\;\color{cyan}011000\; \color{olive}10 \;\color{cyan} 001000\\
\end{align*}
Wandelt man die \color{cyan}cyan \color{black} markierten Daten in Hex. Darstellung um, erhält man den Unicode Character U+1F608.

\section{Bitinterpretation}

\paragraph{(a)}
Umwandeln von $0x447b7d00$ in die binäre Darstellung. Hierbei markiert sind \color{green} Vorzeichen \color{black},\color{blue} Exponent\color{black},\color{red} Mantisse.
\begin{align*}
    \color{green}0\color{blue}100\;0100\;0\color{red}111\;1011\;0111\;1101\;0000\;0000\;
\end{align*}
\color{black}
Vorzeichen lässt sich direkt ablesen $\rightarrow$ Zahl ist positiv\\
Exponent ist gegeben durch $0\text{b}10001000=128+8=136$. Mit dem Bias ergibt sich dann $e=9$.\\ Für die Mantisse erhält man durch die Stufenzahlendarstellung den folgenden Wert
\begin{align*}
    m_2=1111\;0110\;1111\;1010_2=\frac{1}{2}+\frac{1}{4}+\frac{1}{8}+\frac{1}{16}+\frac{1}{64}+\frac{1}{128}+\frac{1}{512}++\frac{1}{1024}+\frac{1}{2048}\\+\frac{1}{4096}+\frac{1}{8192}+\frac{1}{32768}=0.9647521973_{10}
\end{align*}
Dann erhält man für die Gleitkommazahl den folgenden Wert
\begin{align*}
    z_{IEEE754}=(-1)^0\cdot 1.9647521973_10*2^{136-127}=1005,953125_{10}
\end{align*}


\paragraph{(b)}
Die hexadezimal Darstellung der ersten 16bit Integer ist $0\text{x}447b$. Umwandlung in binäre liefert dann $0\text{b}0100\;0100\;0111\;1011$. Mit dem Stufenzahlenverfahren erhält man dann einen Wert von $z_{16bit,1}=17531$.\\
Für die zweite Zahl ergibt sich dann $0\text{x}7d00=0\text{b}0111\;1101\;0000\;0000=32000_{10}$


\paragraph{(c)}
Interpretiert man $0\text{0}447b7d00$ als ASCII bzw. UTF-8 erhält man die folgende Darstellung.\\
In ASCII entspricht $0\text{x}44=D$, $0\text{x}7b=\{$, $0\text{x}7d=\}$ und $0\text{x}00=\text{\textbackslash}0$.\\
Interpretiert man $0\text{0}447b7d00$ in UTF-8 erhält man das gleiche Ergenis wie in ASCII, da UTF-8 die ersten 128 Zeichen aus dem ASCII-Code übernommen hat.

\end{document}
