\documentclass[a4paper]{article}
\usepackage[utf8]{inputenc}
\usepackage{amssymb}
\usepackage[ngerman]{babel}
\usepackage{hyperref}
\usepackage{enumitem}
\usepackage{listings}
\usepackage{esvect}
\usepackage{float}
\usepackage{graphicx}
\usepackage{todonotes}
\usepackage{pgfplots}
\usepackage{verbatim}
\pgfplotsset{compat=1.10}
\usepgfplotslibrary{fillbetween}
\usetikzlibrary{patterns}
\usepackage{mathtools}
\usepackage{centernot}

\hypersetup{
     colorlinks   = true,
     citecolor    = gray
}
\title{Grundlagen der Betriebssysteme\\ Blatt 01 \\ Gruppe 055}
\author{Marco Deuscher & Ibrahem Hasan}
\date{Mai 2019}

\begin{document}

\maketitle

\section{Festkomma Darstellung}
\paragraph{(a)}
\begin{align*}
    7,75_{10}=4+2+1+\frac{1}{2}+\frac{1}{4}=00111110_2
\end{align*}
Da die Zahl aus zweier Potenzen zusammengesetzt werden kann, ist sie exakt darstellbar.


\paragraph{(b)}
\begin{align*}
    2,71_{10}\approx2+\frac{1}{2}+\frac{1}{4}=00010110_2=2.75_{10}
\end{align*}
Die Zahl ist nicht exakt darstellbar. $|Z_{orginal}-Z_{Umwandlung}|=0,04$


\paragraph{(c)}
\begin{align*}
    5,375_{10} = 4+1+\frac{1}{4}+\frac{1}{8}=00101011_2
\end{align*}
Da die Zahl aus zweier Potenzen zusammengesetzt werden kann, ist sie exakt darstellbar.

\paragraph{(d)}
\begin{align*}
    9,12_{10}\approx=01001001_2=9,125_{10}
\end{align*}
Die Zahl ist nicht exakt darstellbar. $|Z_{orginal}-Z_{Umwandlung}|=0,005$





\end{document}
