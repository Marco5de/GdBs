\documentclass[a4paper]{article}
\usepackage[utf8]{inputenc}
\usepackage{amssymb}
\usepackage[ngerman]{babel}
\usepackage{hyperref}
\usepackage{enumitem}
\usepackage{listings}
\usepackage{esvect}
\usepackage{float}
\usepackage{graphicx}
\usepackage[table]{xcolor}% http://ctan.org/pkg/xcolor
\usepackage{todonotes}
\usepackage{pgfplots}
\usepackage{verbatim}
\usepackage{multirow}
\usepackage{booktabs}
\pgfplotsset{compat=1.10}
\usepgfplotslibrary{fillbetween}
\usetikzlibrary{patterns}
\usepackage{mathtools}
\usepackage{centernot}

\hypersetup{
     colorlinks   = true,
     citecolor    = gray
}
\title{Grundlagen der Betriebssysteme\\ Blatt 11 \\ Gruppe 055}
\author{Marco Deuscher & Ibrahem Hasan}
\date{Juli 2019}

\begin{document}

\maketitle
\section{Seitenadressierung}

\begin{enumerate}
    \item Addition des Segmenttabellenbasisregisters und der logischen Segmentnummer der logischen Adresse ergibt Adresse des Segmenteintrags. (Software)
    \item Lesen des zugehörigen Segmenteintrags, der die Startadresse der zugehörigen Seiten-Kachel-Tabelle (SKT) enthält.(Software)
    \item Berechnen der Adresse des SKT-Eintrags: Addition der Startaddresse der SKT mit der SKT-Nummer welche Teil der virtuellen Adresse ist (Software)
    \item Vergleich der Seitennummer mit Segmentlänge. Falls die Seitennummer außerhalb des Segments: es wird eine Unterbrechung ausgelöst (Software)
    \item Es wird der referenzierte Eintrag aus der SKT geladen (Hardware)
    \item Ermitteln des Präsenzbits. Dies ist in diesem Fall: 0, da Seite nicht geladen (Hardware)
    \item Es wird eine Unterbrechung ausgelöst, da ein Pagefault aufgetreten ist (Hardware)
    \item Blockieren des Prozesses, einlagern der benötigten Seite in die freie Kachel (Software).
    \item Anpassen des SKT Eintrags (setzen des Referenzbits und anpassen der Adresse) (Hardware) 
    \item Einlagern der Seite beendet, Prozess wird wieder aufgeweckt. Der lesende Speicherzugriff der Anwendung wird wiederholt (Software)
\end{enumerate}

\newpage

\section{Ersetzungsstrategien}
\begin{table}[h]
    \centering
    \begin{tabular}{c c|c|c|c|c|c|c|c|c|c|c}
        \toprule
         Referenzfolge & & 1 & 2 & 3 & 1 & 2 & 4 & 5 & 1 & 2 & 3  \\
         \midrule
         \multirow{3}{*}{Hauptspeicher} & 
         Kachel 1 & 1 & 1 & 1 & 1 & 1 & 1 & 5 & 5 & 5 & 3 \\ 
          & Kachel 2 & - & 2 & 2 & 2 & 2 & 2 & 2 & 1 & 1 & 1 \\ 
           & Kachel 3 &- & - & 3 & 3 & 3 & 4 & 4 & 4 & 2 & 2\\ 
           \midrule
           \multirow{3}{*}{Kontrollzustände / Referenzbits} & 
           Kachel 1 & 0 & 1 & 2 & 0 & 1 & 2 & 0 & 1 & 2 & 0\\ 
          & Kachel 2 & $>$& 0 & 1 & 2 & 0 & 1 & 2 & 0 & 1 & 2 \\ 
           & Kachel 3 & $>$& $>$ & 0 & 1 & 2 & 0 & 1 & 2 & 0 & 1\\ 
           \bottomrule
    \end{tabular}
    \caption{Least Recently Used mit 5+3 Seitenersetzungen}
    \label{tab:my_label}
\end{table}


\begin{table}[h]
    \centering
    \begin{tabular}{c c|c|c|c|c|c|c|c|c|c|c}
        \toprule
         Referenzfolge & & 1 & 2 & 3 & 1 & 2 & 4 & 5 & 1 & 2 & 3  \\
         \midrule
         \multirow{3}{*}{Hauptspeicher} & 
         Kachel 1 & 1 & 1 & 1 & 1 & 1 & 4 & 4 & 4  & 2 & 2 \\ 
          & Kachel 2 & - & 2 & 2 & 2 & 2 & 2 & 5 & 5 & 5 & 3  \\ 
           & Kachel 3 & - & - & 3 & 3 & 3 & 3 & 3 & 1 & 1 & 1\\ 
           \midrule
           \multirow{3}{*}{Kontrollzustände / Referenzbits} & 
           Kachel 1 & 0 & 1 & 2 & 3 & 4 & 0 & 1 & 2 & 0 & 1\\ 
          & Kachel 2 &$>$ & 0 & 1 & 2 & 3 & 4  & 0 & 1 & 2 & 0 \\ 
           & Kachel 3 & $>$ & $>$ & 0 & 1 & 2 & 3  & 4 & 0 & 1 & 2 \\ 
          
           \bottomrule
    \end{tabular}
    \caption{FIFO mit 5+3 Seitenersetzungen}
    \label{tab:my_label}
\end{table}


\begin{table}[h]
    \centering
    \begin{tabular}{c c|c|c|c|c|c|c|c|c|c|c}
        \toprule
         Referenzfolge & & 1 & 2 & 3 & 1 & 2 & 4 & 5 & 1 & 2 & 3  \\
         \midrule
         \multirow{3}{*}{Hauptspeicher} &
         Kachel 1 & 1 & 1 & 1 & 1 & 1 & 4 & 4 & 4 & 2 & 2 \\ 
          & Kachel 2 & - & 2 & 2 & 2 & 2 & 2 & 5 & 5 &5 & 3 \\ 
           & Kachel 3 & - & - & 3 & 3 & 3 & 3 & 3 & 1& 1& 1\\ 
           \midrule
           \multirow{4}{*}{Kontrollzustände / Referenzbits} & 
           Kachel 1 & 1 & 1 & 1 & 1 & 1 & 1 & 1 & 1& 1 &1 \\ 
          & Kachel 2 & 0 & 1 & 1 & 1 & 1 & 0& 1 & 1& 0& 1\\ 
           & Kachel 3 & 0 & 0 & 1 & 1 & 1 & 0& 0 & 1& 0& 0\\ 
           & Umlaufzeiger & 2 & 3 & 1 & 1 & 1 & 2 & 3 & 1 & 2 & 3 \\ 
           \bottomrule
    \end{tabular}
    \caption{Second Chance Clock mit 5+3 Seitenersetzungen}
    \label{tab:my_label}
\end{table}




\end{document}