\documentclass[a4paper]{article}
\usepackage[utf8]{inputenc}
\usepackage{amssymb}
\usepackage[ngerman]{babel}
\usepackage{hyperref}
\usepackage{enumitem}
\usepackage{listings}
\usepackage{esvect}
\usepackage{float}
\usepackage{graphicx}
\usepackage{xcolor}
\usepackage{todonotes}
\usepackage{pgfplots}
\usepackage{verbatim}
\pgfplotsset{compat=1.10}
\usepgfplotslibrary{fillbetween}
\usetikzlibrary{patterns}
\usepackage{mathtools}
\usepackage{centernot}

\hypersetup{
     colorlinks   = true,
     citecolor    = gray
}
\title{Grundlagen der Betriebssysteme\\ Blatt 08 \\ Gruppe 055}
\author{Marco Deuscher & Ibrahem Hasan}
\date{Juni 2019}

\begin{document}

\maketitle

\section{Dateisysteme: FAT}
\paragraph{(a)}
Die erste Zeile enthält die jeweilige Clusternummer, während die zweite Zeile Informationen über das jeweilige Cluster enthält.\\
Bspw. $F7_{16}$ für defekt oder $00_{16}$ für frei.

\paragraph{(b)}
\begin{itemize}
    \item Datei B: $02_{16}$ - $04_{16}$ - $03_{16}$
    \item Datei C: $05_{16}$ - $06_{16}$ - $09_{16}$
    \item Datei D: $07_{16}$
    \item Datei E: $0B_{16}$ - $0F_{16}$ 
    \item Datei F: $0C_{16}$ - $0D_{16}$ - $0E_{16}$ - $10_{16}$ - $11_{16}$
\end{itemize}

\paragraph{(c)}
\begin{itemize}
    \item Datei 1 = Datei D. Belegt ein Cluster\\
        min. 1Byte - max. 8KByte
    \item Datei 2 = Datei B oder C. Belegt drei Cluster\\
        min. 16KByte + 1Byte - max. 24KByte
    \item Datei 3 = Datei B oder C. Belegt drei Cluster\\
         min. 16KByte + 1Byte - max. 24KByte
    \item Datei 4 = Datei E. Belegt zwei Cluster\\
        min. 8KByte + 1 Byte - max. 16KByte
    \item Datei 5 = Datei F. Belegt fünf Cluster\\
        min. 40KByte + 1Byte - max. 48KByte\\
        Diese Datei sollte eigentlich 6 Cluster belegen, es wäre möglich dass zu dieser Datei einer der defekten Blöcke gehört.
\end{itemize}


\section{Dateisysteme: NTFS}
\paragraph{Datei 1: hallo.txt 100 Byte lang}
\begin{itemize}
    \item Standard Informationen
        \begin{itemize}
            \item Länge: 100Bytes
            \item MS-DOS Atrribute
            \item Zeitstempel
            \item Anzahl der Hard-Links
            \item Sequenznummer der gültigen File Reference
        \end{itemize}
    \item Dateiname: \textit{hallo.txt}
    \item Zugriffsrechte
    \item Daten
\end{itemize}

\paragraph{rsa.key 8 KByte lang}
\begin{itemize}
    \item Standard Informationen
        \begin{itemize}
            \item Länge: 8KBytes
            \item MS-DOS Atrribute
            \item Zeitstempel
            \item Anzahl der Hard-Links
            \item Sequenznummer der gültigen File Reference
        \end{itemize}
    \item Dateiname: \textit{rsa.key}
    \item Zugriffsrechte
    \item Daten
\end{itemize}

\paragraph{cat.jpg 44 Kbyte lang}
\begin{itemize}
    \item Standard Informationen
        \begin{itemize}
            \item Länge: 44KBytes
            \item MS-DOS Atrribute
            \item Zeitstempel
            \item Anzahl der Hard-Links
            \item Sequenznummer der gültigen File Reference
        \end{itemize}
    \item Dateiname: \textit{cat.jpg}
    \item Zugriffsrechte
    \item Verweis auf \\
        \begin{itemize}
            \item VCN 0 - LCN 17 - Anzahl Cluster 1
            \item VCN 1 - LCN 19 - Anzahl Cluster 1
            \item VCN 2 - LCN 22 - Anzahl Cluster 9
        \end{itemize}
        $\rightarrow$ es werden insgesamt 11 Cluster benötigt
    \end{itemize}





\end{document}