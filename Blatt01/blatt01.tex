\documentclass[a4paper]{article}
\usepackage[utf8]{inputenc}
\usepackage{amssymb}
\usepackage[ngerman]{babel}
\usepackage{hyperref}
\usepackage{enumitem}
\usepackage{listings}
\usepackage{esvect}
\usepackage{float}
\usepackage{graphicx}
\usepackage{todonotes}
\usepackage{pgfplots}
\usepackage{verbatim}
\pgfplotsset{compat=1.10}
\usepgfplotslibrary{fillbetween}
\usetikzlibrary{patterns}
\usepackage{mathtools}
\usepackage{centernot}

\hypersetup{
     colorlinks   = true,
     citecolor    = gray
}
\title{Grundlagen der Betriebssysteme\\ Blatt 01 \\ Gruppe 055}
\author{Marco Deuscher}
\date{April 2019}

\begin{document}

\maketitle

\section{Zahlen Konvertierung}
\paragraph{(a)}
\begin{align*}
    89_{10} \rightarrow x_2\\
    89\div2 &= 44 \quad R=1\\
    44\div2&=22 \quad R=0\\
    22\div2&=11 \quad R=0\\
    11\div2&=5 \quad R=1\\
    5\div2&=2 \quad R=1\\
    2\div2&=1 \quad R=0\\
    1\div2&=0 \quad R=1\\
    \Rightarrow x_2&=1011001
\end{align*}

\paragraph{(b)}
\begin{align*}
    32_7 \rightarrow x_5\text{ Division im Dezimalsystem einfacher}\\
    3*7^1+2*7^0&=23_{10}\\
    23\div5&=4\quad R=3\\
    4\div5&=0\quad R=4\\
    \Rightarrow x_5&=43\\
\end{align*}

\paragraph{(c)}
\begin{align*}
    4360_{10}\rightarrow x_2\\
    4360=4096+256+8=2^{12}+2^8+2^3\\
    &1000\;0000\;0000\\
    +&0000\;1000\;0000\\
    +&0000\;0000\;0100\\
    =&1000\;1000\;0100\\
\end{align*}

\paragraph{(d)}
\begin{align*}
    1414215376_8 \rightarrow x_2 \rightarrow x_{16}\\
    x_2 = 001\;100\;001\;100\;010\;001\;101\;011\;111\;110\\
    00\;1100\;0011\;0001\;0001\;1010\;1111\;1110\;\\
    \Rightarrow x_{16}=C311AFE
\end{align*}


\section{Zahlen Konvertierung II}
\paragraph{(a)}
\begin{align*}
    CAFFEE_{16}\rightarrow x_8\\
    1100\;1010\;1111\;1111\;1110\;1110\\
    110\;010\;101\;111\;111\;111\;101\;110\\
    \Rightarrow x_8 = 62577756_8
\end{align*}

\paragraph{(b)}
\begin{align*}
    3072_{10}\rightarrow x_2\\
    3072=2048+1024=2^{11}+2^{10}\\
    &0100\;0000\;0000\\
    +&0010\;0000\;0000\\
    =&0110\;0000\;0000\\
\end{align*}

\paragraph{(c)}
\begin{align*}
    1724656_8 \rightarrow x_{32}\\
    \text{$32=2^5$}\rightarrow \text{5bit pro Zahl}\\
    &001\;111\;010\;100\;110\;101\;110\\
    =&0\;01111\;01010\;01101\;01110 = FADE\\
\end{align*}

\paragraph{(d)}
\begin{align*}
    1316_{10}\rightarrow x_2\\
    1316 = 1024+256+32+4=2^{10}+2^8+2^5+2^2\\
    &10\;0000\;0000\\
    +&00\;1000\;0000\\
    +&00\;0001\;0000\\
    +&00\;0000\;0010\\
    =&10\;1001\;0010\\
\end{align*}

\section{Binäre Addition}
\paragraph{(a)}
\begin{align*}
    &101\;1001\\
    +&001\;1000\\
    +&011\;0000\\
    =&111\;0001\\
\end{align*}

\paragraph{(b)}
\begin{align*}
    &0010\;1101\\
    +&0101\;1111\\
    +&1111\;1110\\
    =&1000\;1100\\
\end{align*}

\paragraph{(c)}
\begin{align*}
    &0100\;1100\\
    +&0110\;0010\\
    +&0100\;0000\\
    =&1010\;1110\\
\end{align*}

\paragraph{(d)}
\begin{align*}
    &0010\;1011\\
    +&0011\;0111\\
    +&0111\;1110\\
    =&0110\;0010\\
\end{align*}


\section{Komplementbildung}
\paragraph{(a)}
\begin{align*}
    2018_{10}&=0000\;0111\;1110\;0010\\
    \text{Invertieren liefert dann}\\
    ~0000\;0111\;1110\;0010 &= 1111\;1000\;0001\;1101\\
    \text{+1:}\\
    -2018_{10} &= 1111\;1000\;0001\;1110_2\\
\end{align*}

\paragraph{(b)}
\begin{align*}
    27347_{10}&=0110\;1010\;1101\;0011_2\\
    \text{Invertieren liefert dann}\\
    ~0110\;1010\;1101\;0011 &= 1001\;0101\;0010\;1100\\
    \text{+1:}\\
    -27347_{10} &= 1001\;0101\;0010\;1101\\
\end{align*}

\section{Binäre Multiplikation}
\paragraph{(a)}
\begin{align*}
    &010110*111=10011010\\
    &01011000\\
    +&00101100\\
    +&00010110\\
    +&01111000\\
    =&10011010\\
\end{align*}

\paragraph{(b)}
\begin{align*}
    &10010010*1001001=10\;1001\;1010\;0010\\
    &10\;0100\;1000\;0000\\
    +&00\;0000\;0000\;0000\\
    +&00\;0000\;0000\;0000\\
    +&00\;0100\;1001\;0000\\
    +&00\;0000\;0000\;0000\\
    +&00\;0000\;0000\;0000\\
    +&00\;0001\;1001\;0010\\
    =&10\;1001\;1010\;0010\\
\end{align*}


\paragraph{(c)}
\begin{align*}
    &10011110*10101=1100\;1111\;0110\\
    &1001\;1110\;0000\\
    +&0000\;0000\;0000\\
    +&0010\;0111\;1000\\
    +&0000\;0000\;0000\\
    +&0000\;1001\;1110\\
    =&1100\;1111\;0110\\
\end{align*}
\end{document}

