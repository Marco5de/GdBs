\documentclass[a4paper]{article}
\usepackage[utf8]{inputenc}
\usepackage{amssymb}
\usepackage[ngerman]{babel}
\usepackage{hyperref}
\usepackage{enumitem}
\usepackage{listings}
\usepackage{esvect}
\usepackage{float}
\usepackage{graphicx}
\usepackage{xcolor}
\usepackage{todonotes}
\usepackage{pgfplots}
\usepackage{verbatim}
\pgfplotsset{compat=1.10}
\usepgfplotslibrary{fillbetween}
\usetikzlibrary{patterns}
\usepackage{mathtools}
\usepackage{centernot}

\hypersetup{
     colorlinks   = true,
     citecolor    = gray
}
\title{Grundlagen der Betriebssysteme\\ Blatt 07 \\ Gruppe 055}
\author{Marco Deuscher & Ibrahem Hasan}
\date{Juni 2019}

\begin{document}

\maketitle

\section{Deadlocks}
\paragraph{(a)}
Ein Szenario in dem keines der Fahrzeuge die Kreuzung überqueren kann, ergibt sich wenn alle vier Fahrzeuge auf die erste Spur vorfahren. Dann wird jedes der Fahrzeuge durch ein anderes blockiert, so dass alle Fahrzeuge warten.\\

\paragraph{(b)}
\todo[inline]{wie ist dieser Fall zu verstehen}

\paragraph{(c)}
\begin{itemize}
    \item Fahrzeuge entsprechen Aktivitätsträger
    \item Kreuzung entsprechen geteilten Ressourcen
    \item Warten entspricht blockiert sein
    \item Nie überqueren entspricht dem starvation Fall, in dem ein Akitvitätsträger dauerhaft blockiert ist und nie mehr zum laufen kommt
\end{itemize}

\paragraph{(d)}
Wenn ein Auto fahren möchte, fährt es bis zur ersten Spur nach vorne und versucht ob es die Kreuzung überqueren kann. Ist dies möglich, so überquert das Auto die Kreuzung. Kann es nicht fahren, weil die zweite Spur blockiert ist, fährt er wieder zurück, bis er wieder in seiner Haltebucht steht. Dort wartet er dann eine zufällig lang gewählte Zeit bevor er es erneut versucht.


\section{Semaphore und aktives Warten (Spinlock)}
\begin{itemize}
    \item ein Semaphor ruft in der Regel yield auf, so dass der Aktivitätsträger nicht die CPU belastetet. Dies ist ein Vorteil, da dann diese CPU Zeit für Prozesse genutzt werden kann, so dass der blockierte Aktvitätsträger schneller auf die Ressource zugreifen kann
    \item ein Spinlock implementiert busy-waiting, so dass dieser die CPU nicht abgibt. Ein Spinlock kann bspw. in einer ISR verwendet werden, da dort der Aktivitätsträger nicht schlafen gelegt werden kann
\end{itemize}

\paragraph{(b)}
Spinlocks eignen sich nicht auf Einprozessorgeräten, da diese die CPU blockieren. Auf einem System mit nur einem Kern, verliert man damit effektiv nur Rechenzeit und hat keinen Vorteil.


\paragraph{(c)}
Die Operationen P und V selbst müssen nicht atomar sein, allerdings muss das erhöhen oder erniedrigen des Semaphors atomar ablaufen. Dafür benötigt man eine Instruktion welche innerhalb eines Cycles eine Speicherzelle lesen und schreiben kann.

\end{document}