\documentclass[a4paper]{article}
\usepackage[utf8]{inputenc}
\usepackage{amssymb}
\usepackage[ngerman]{babel}
\usepackage{hyperref}
\usepackage{enumitem}
\usepackage{listings}
\usepackage{esvect}
\usepackage{float}
\usepackage{graphicx}
\usepackage{xcolor}
\usepackage{todonotes}
\usepackage{pgfplots}
\usepackage{verbatim}
\pgfplotsset{compat=1.10}
\usepgfplotslibrary{fillbetween}
\usetikzlibrary{patterns}
\usepackage{mathtools}
\usepackage{centernot}

\usepackage{tikz-timing}[2009/05/15]
\pagestyle{empty}
\usepackage{subcaption}

\hypersetup{
     colorlinks   = true,
     citecolor    = gray
}
\title{Grundlagen der Betriebssysteme\\ Blatt 04 \\ Gruppe 055}
\author{Marco Deuscher & Ibrahem Hasan}
\date{Mai 2019}


\begin{document}
\maketitle

\section{Prozesse und Prozessverwaltung}
\paragraph{(a)}
\begin{enumerate}
    \item Neu (erzeugt)
    \item Bereit
    \item Laufend
    \item blockiert
    \item terminiert
\end{enumerate}

\paragraph{(b)}
\begin{itemize}
    \item Neu $\rightarrow$ Bereit: neu erstellter Prozess kann zugelassen werden
    \item Bereit $\rightarrow$ Laufend: Prozess der bereit ist, kann CPU Zeit bekommen
    \item Laufend $\rightarrow$ Bereit: Laufender Prozess kann Unterbrochen werden und wieder in bereit eingereiht werden
    \item Laufend $\rightarrow$ Blockiert: laufender Prozess blockiert implizit oder explizit
    \item Blockiert $\rightarrow$ Bereit: nach Freigabe einer Ressource kann zuvor blockierter Prozess wieder laufen
\end{itemize}

\section{Shortest Job First Scheduling}

\paragraph{(a)}
Shortest Job First optimiert die mittlere Wartezeit. Es werden so viele Prozesse wie möglich in den blockiert/terminiert Status gesetzt, da zuerst kurze Prozesse abgearbeitet werden.

\paragraph{(b)}
Wenn kontinuierlich neue kurze Prozesse hinzukommen, bekommen rechenintensive längere Prozesse keine CPU-Zeit mehr, da immer die kürzeren Prozesse gewählt werden.

\paragraph{(c)}
Beim präemptiven Shortest Job First kann der ausgeführt Task unterbrochen werden (Context Switch). \\
Beim nicht-präemptiven Shortest Job First ist dies nicht möglich, so dass der ausgewählte Task so lange läuft, bis dieser blockiert oder terminiert.


\begin{figure}[tbh]
    %\centering
    \begin{subfigure}[b]{0.3\textwidth}
        \centering
        \begin{tikztimingtable}
          Laufend   & 2H 2L 2H 2L 4H 2L 6L 8H\\ 
          Bereit  & 2L 2H 2L 2H 20L \\
          Blockiert & 12L 8H 8L \\
          \begin{extracode}
                        \vertlines[help lines]{0, 4, ..., \twidth}
                        \foreach \x in {0, ...,7}
                            \node at (\x*4,2.5) {\x s};
                    \end{extracode}
        \end{tikztimingtable}
        \caption{PSJF Prozess A}
    \end{subfigure}
    
    \begin{subfigure}[b]{0.3\textwidth}
        \centering
        \begin{tikztimingtable}
          Laufend   & 2L 2H 2L 2H 20L\\ 
          Bereit  & 28L\\
          Blockiert &  4L 2H 22L\\
          \begin{extracode}
                        \vertlines[help lines]{0, 4, ..., \twidth}
                        \foreach \x in {0, ...,7}
                            \node at (\x*4,2.5) {\x s};
                    \end{extracode}
        \end{tikztimingtable}
        \caption{PSJF Prozess B}
    \end{subfigure}
    
    \begin{subfigure}[b]{0.3\textwidth}
        \centering
        \begin{tikztimingtable}
          Laufend   & 12L 8H 8L\\
          Bereit  & 2L 10H 16L\\ 
          Blockiert &  28L\\
          \begin{extracode}
                        \vertlines[help lines]{0, 4, ..., \twidth}
                        \foreach \x in {0, ...,7}
                            \node at (\x*4,2.5) {\x s};
                    \end{extracode}
        \end{tikztimingtable}
        \caption{PSJF Prozess C}
    \end{subfigure}
    \caption{PSJF}
    \label{fig:my_label}
\end{figure}

\begin{figure}[tbh]
    %\centering
    \begin{subfigure}[b]{0.3\textwidth}
        \centering
        \begin{tikztimingtable}
          Laufend   & 8H 12L 8H\\ 
          Bereit  & 16L 4H 8L\\
          Blockiert & 8L 8H 10L\\
          \begin{extracode}
                        \vertlines[help lines]{0, 4, ..., \twidth}
                        \foreach \x in {0, ...,7}
                            \node at (\x*4,2.5) {\x s};
                    \end{extracode}
        \end{tikztimingtable}
        \caption{SJF Prozess A}
    \end{subfigure}
    
    \begin{subfigure}[b]{0.3\textwidth}
        \centering
        \begin{tikztimingtable}
          Laufend   & 8L 2H 8L 2H 8L\\ 
          Bereit  & 2L 6H 4L 6H 10L\\
          Blockiert & 10L 2H 16L\\
          \begin{extracode}
                        \vertlines[help lines]{0, 4, ..., \twidth}
                        \foreach \x in {0, ...,7}
                            \node at (\x*4,2.5) {\x s};
                    \end{extracode}
        \end{tikztimingtable}
        \caption{SJF Prozess B}
    \end{subfigure}
    
    \begin{subfigure}[b]{0.3\textwidth}
        \centering
        \begin{tikztimingtable}
          Laufend   & 10L 8H 10L\\
          Bereit  & 2L 8H 18L\\ 
          Blockiert &  28L\\
          \begin{extracode}
                        \vertlines[help lines]{0, 4, ..., \twidth}
                        \foreach \x in {0, ...,7}
                            \node at (\x*4,2.5) {\x s};
                    \end{extracode}
        \end{tikztimingtable}
        \caption{SJF Prozess C}
    \end{subfigure}
    \caption{SJF}
    \label{fig:my_label}
\end{figure}
    

\end{document}

